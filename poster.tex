% !TeX TS-program = lualatex
% vim: set spelllang=fr:
\documentclass{poster}
\setmainlanguage{french}

\title{Série \emph{\enquote{Pionnières de l’informatique}}}
\author{AÉDIROUM}
\date{février~2023}

\begin{document}\hspace*{-15pt}%
    %%%%%%%%%%%%%%%%%%%%%%%%%%%%%%%%%%%%%%%%%%%%%%%%%%%%%%%%%%%%%%%%%%%%%%%%%%%
    % KATHERINE JOHNSON
    \begin{tikzpicture}
        \definecolor{background tint}{HTML}{9da2cc}
        \definecolor{text accent}{HTML}{e2ecfa}
        \definecolor{text main}{HTML}{f4f6fa}

        \background{images/johnson}
        \name{Katherine}{Johnson}
            {right}{south east}{([xshift=5pt]page cs:.95,.54)}

        \biography
            {(page cs:.2,.49) .. controls (page cs:.25,.125)
                and (page cs:.62,.331) .. (page cs:.62,-.1)}
            {(page cs:.95,.49) -- (page cs:.95,-.1)}{
                \textbf{Katherine Goble Johnson} (1918--2020) était une
                mathématicienne, physicienne et ingénieure spatiale. Pendant
                plus de trente ans, elle a réalisé pour la NASA les calculs
                de trajectoire complexes nécessaires à des
                missions telles que \emph{Freedom~7} (première mission spatiale
                étasunienne habitée en~1961) et \emph{Apollo~11} (premiers pas
                sur la Lune en~1969). Plus tard, elle a contribué à la
                transition vers l’utilisation d’ordinateurs pour réaliser ces
                calculs.
            }

        \notice{%
            \enquote{Katherine Johnson à son bureau
                du \emph{Langley Research Center}}~(septembre~1966). NASA.
        }{%
            Tirés de Wikipédia en date du 11~février~2023.
        }
    \end{tikzpicture}
    %%%%%%%%%%%%%%%%%%%%%%%%%%%%%%%%%%%%%%%%%%%%%%%%%%%%%%%%%%%%%%%%%%%%%%%%%%%
    % GRACE HOPPER
    \begin{tikzpicture}
        \definecolor{background tint}{HTML}{c78a69}
        \definecolor{text accent}{HTML}{ffd2ba}
        \definecolor{text main}{HTML}{ffefe7}

        \background{images/hopper}
        \fill[black, opacity=.75] (page cs:-.85,-.5) rectangle (page cs:-.2,-.25);
        \name{Grace}{Hopper}
            {left}{south west}{([xshift=-5pt]page cs:-.815,-.45)}
        \biography
            {(page cs:-.815,-.5) -- (page cs:-.815,-.9)}
            {(page cs:-.015,-.5) -- (page cs:-.08,-.9)}{
                \textbf{Grace Murray Hopper} (1906--1992) était une
                informaticienne et mathématicienne, l’une des premières à proposer
                l’idée de langages de programmation indépendants des machines.
                Elle a développé le premier compilateur, le \emph{A-0~System},
                puis le langage \emph{COBOL}. Elle a aussi contribué au
                développement et à la programmation des \emph{Harvard Mark~I} à
                \emph{III}. Le \emph{bug} informatique le plus emblématique, un
                insecte coincé dans un relais du \emph{Mark~II} en~1947, 
                fût découvert par son équipe.
            }

        \notice{%
            \enquote{Grace Hopper examinant le mécanisme de séquençage du
                \emph{Harvard~Mark~I}}~(août~1944). \emph{National Museum of
                American History, Archives Center} (NMAH.AC.0324.36).
        }{%
            Tirés de Wikipédia en date du 11~février~2023.
        }
    \end{tikzpicture}
    %%%%%%%%%%%%%%%%%%%%%%%%%%%%%%%%%%%%%%%%%%%%%%%%%%%%%%%%%%%%%%%%%%%%%%%%%%%
    % MARGARET HAMILTON
    \begin{tikzpicture}
        \definecolor{background tint}{HTML}{7b8f5b}
        \definecolor{text accent}{HTML}{e3f4c6}
        \definecolor{text main}{HTML}{f7f8f4}

        \background{images/hamilton}
        \name{Margaret}{Hamilton}{right}{north east}{([xshift=-5pt]page cs:.75,.27)}
        \biography
            {(page cs:-.86,.12) -- (page cs:-.86,-.9)}
            {(page cs:-.37,.12) -- (page cs:-.4,0) -- (page cs:-.4,-.9)}{
                \textbf{Margaret Heafield Hamilton} (1936-) est une
                informaticienne et ingénieure en logiciel. Elle a travaillé sur
                le développement du logiciel embarqué des programmes
                \emph{Apollo} et \emph{Skylab}, en particulier la mission
                \emph{Apollo~11} qui a vu l’humain marcher sur la Lune en 1969.
                Elle est à l’origine du terme \enquote{génie logiciel} et de
                nombre de ses principes fondamentaux. Hamilton a contribué à
                l’élever au même rang que les autres disciplines de génie.
            }

        \notice{%
            \enquote{Margaret Hamilton posant avec le code source du logiciel
                embarqué du programme Apollo}~(1969). \emph{MIT Museum,} restaurée par Adam Cuerden.
        }{%
            Tirés de Wikipédia en date du 11~février~2023.
        }
    \end{tikzpicture}
    %%%%%%%%%%%%%%%%%%%%%%%%%%%%%%%%%%%%%%%%%%%%%%%%%%%%%%%%%%%%%%%%%%%%%%%%%%%
    % ADA LOVELACE
    \begin{tikzpicture}
        \definecolor{background tint}{HTML}{8c882f}
        \definecolor{text accent}{HTML}{fffdb0}
        \definecolor{text main}{HTML}{fffeed}

        \background{images/lovelace}
        \name{Ada}{Lovelace}{left}{south west}{([xshift=-5pt]page cs:-.875,.27)}
        \biography
            {(page cs:-.875,.22) -- (page cs:-.875,.01)
                -- (page cs:-.485,-.9)}
            {(page cs:0,.22) .. controls (page cs:-.3,0)
                .. (page cs:-.4,-.3) -- (page cs:-.4,-.9)}{
                \textbf{Augusta Ada King, comtesse de Lovelace} (1815--1852)
                était une mathématicienne, considérée comme la première
                programmeuse de l’histoire. Dans un mémoire de 1843 sur la
                \emph{machine analytique} de Babbage, elle décrit un programme
                qui calcule les nombres de Bernoulli, comprenant la première
                occurrence connue d’une boucle \enquote{tant que}. Lovelace
                anticipe aussi le potentiel des machines à manipuler non
                seulement des nombres, mais plus généralement toute forme de
                données.
            }

        \notice{%
            \enquote{Portrait d’Ada Lovelace par Alfred Edward Chalon}~(1838).
            \emph{Computer History Museum.}
        }{%
            Tirés de Wikipédia en date du 11~février~2023.
        }
    \end{tikzpicture}
\end{document}